\chapter{Sprites}

\section{Introduction}

Sprites are graphic images of differing sizes (8x8, 16x16, 24x24 or 32x32) which can appear on top of a bitmap but which don't affect that bitmap. It is a bit like a cartoon where you have a background and animated characters are placed on it.

The sprite command adopts the same syntax as the graphics commands in the previous section. An example is:

\begin{verbatim}
	sprite 3 image 5 to 20,20
\end{verbatim}

This manipulates sprite number 3 (there are 64, numbered 0..63), using image number 5, centred on screen position 20,20. The image number comes from the data set - in the example set below it would be enemy.png rotated by 90 - the sixth entry as we count from 0. You can change the location and image independently.

Most of the theoretical graphics options do not work ; you cannot colour, scale, flip etc. a sprite, the graphic is what it is. However, they are very fast compared with drawing an image on the screen using the IMAGE command.

\section {Creating sprites}

In SuperBASIC sprite data is loaded, with an index, to memory location \$30000. This chapter explains how to create that index.

\section{Getting images}

Sprite data is built from PNG images up to 32x32. There are some examples in the Solarfox directory in the github https://github.com/paulscottrobson/superbasic.

They can be created individually, or ripped from sprite sheets - this is what ripgfx.py is doing in the Makefile in solarfox/graphics ; starting with the PNG file source .png it is informed where graphics are, and it tries to work out a bounding box for that graphic, and exports it to the various files.

\section{Building a sprite data set}
	
Sprite data sets are built using the spritebuild.py python script (in the utilities subdirectory). Again there is an example of this in the Solarfox directory.

Sprite set building is done using the spritebuild.py script which takes a file of sprite definitions. This is a simple text list of files, which can be either png files as is, or postfixed by a rotate angle (only 0,90,180 and 270) or v or h for vertical and horizontal mirroring.

\begin{verbatim}
graphics/ship.png
graphics/ship.png 90
graphics/ship.png 180
graphics/ship.png 270
graphics/enemy.png
graphics/enemy.png 90
graphics/enemy.png 180
graphics/enemy.png 270
graphics/collect1.png
graphics/collect2.png
graphics/life.png h
\end{verbatim}

Sprite images are numbered in the order they are in the file from zero and should be loaded at \$30000

When building the sprite it strips it as much as possible and centres it in the smallest sprite size it fits in. When using BASIC commands to position a sprite, that position is relative to the centre of the sprite.

\section{Data format}

At present there is a very simple data format. \

\begin{verbatim}
+00 is the format code ($11) 
+01 is the sprite size  (0-3, representing 8,16,24 and 32 pixel size) 
+02 the LUT to use (normally zero) 
+03 the first byte of sprite data 
\end{verbatim}

the size, LUT and data are then repeated for every sprite in the sprite set. The file should end with a sprite size of \$80 (128) to indicate the end of the set.


