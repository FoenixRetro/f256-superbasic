\chapter{Cross Development of BASIC Programs}

Cross development is an alternative to the classic way of programming a home computer, where the programmer types code directly into the machine. Cross development allows you to write the code on a Personal Computer, and upload it through the USB debug port in the F256. It is also possible to do this with machine code and graphic and other data.

\section{Assistance}

In the SuperBasic git, https://github.com/paulscottrobson/superbasic, each release contains a file "howto-crossdev-basic.zip" which gives everything you need to cross develop in BASIC and some example programs.

\section{Connection}

To connect your F256Junior to a PC (Windows, Linux, Mac) you need a standard USB cable with a Micro USB plug. This needs to be a data cable, some cables only provide power. The Micro USB plug plugs into the board, and the USB plug into the PC.

\section{Software}

There are two ways of programming the board. I prefer FnxMgr \emph{https://github.com/pweingar/FoenixMgr} which is a Python script which runs on all platform, and can easily automate uploading. It can also be uploaded through the Foenix IDE in Windows\underline{}.

Besides Python version 3, the FnxMgr script requires pyserial.

\section{BASIC}

The input to the program is standard ASCII files, with line numbers. Line numbers are required for editing only. (There is a python script on the superbasic github which adds these automatically). However, you do not need to use line numbers in programming, though GOTO and GOSUB are implemented if you wish, or want to port old software.

I would start with something simple though.

\example{Print to the screen and make silly sound effect}
\begin{verbatim}
	10 print "Hello, world !"
	20 zap
\end{verbatim}

Each file should end in a character with an ASCII code greater than 127, which marks the end of the file.  You can copy one from the software in github.

\section{Uploading and running}

This is written for people with 'B' boards which automatically start up into BASIC. If you are booting from RAM, or have an A board, it will be slightly different.

Uploading works by loading the ASCII text into memory. It is then effectively 'typed in' by either the \textbf{xload} command or the \textbf{xgo} command. The first loads the program in (and it can then be listed or edited or run in the normal way. The second loads and runs it.

To load the program into memory to be "loaded" you need something like the below. The first one works on my Arch Linux Box. The second is simply a guess ; I do not know what the COM ports are for each system. You should be able to discover this with the Device Manager (Windows) or lsusb (Linux).

\example{Linux Upload}
\begin{verbatim}
	python ../bin/fnxmgr.zip --port /dev/ttyUSB0 --binary load.bas  --address 28000
\end{verbatim}

\example{Windows Upload (not tried)}
\begin{verbatim}
	python ..\bin\fnxmgr.zip --port COM1 --binary load.bas  --address 28000
\end{verbatim}

\section{Memory Use}

Initially the lower 32k of RAM (0000-7FFF) has a logical address equal to its physical address. The BASIC ROM is mapped into 8000-BFFF. 

The memory block C000-DFFF is reserved by the Kernel - you can change I/O registers, but do not map RAM here and change it unless you are absolutely sure of what you are doing.

The memory block E000-FFFF contains the Kernel.

\section{Sprites}
Sprites are loaded (in BASIC) to \$30000 and there is a simple index format. This is covered in the sprites section.
