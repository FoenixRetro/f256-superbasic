\chapter{Graphics}

\section{Introduction}

The graphics subsystem consists of three components, which is a subset of the full capabilities of the F256 machines.\\

Firstly there is an 8x8 pixel tile grid of up to 256x256 tiles, which can be scrolled about.\\

Secondly, on top of that is a 320x240 bitmap screen, which can have anything drawn on it.\\

Thirdly, on top of that, are the sprites.\\

The graphics are much more complex than this ; the system allows up to three tile maps for example. Those can be done in BASIC if you wish, by directly accessing the system registers, as covered in the hardware reference guide.\\

\section{Bitmap Graphics}

Bitmap Graphics can be done in one of three ways. 

\begin{itemize}
\item Firstly, they can be done using BASIC commands like LINE, PLOT and TEXT. These are the easiest.
\item Secondly, they can be done by directly accessing the graphics library via the GFX command. 
\item Thirdly, you can "hit the hardware" directly using POKE and DOKE or the indirection operators.
\end{itemize}
The latter is the most flexible. BASIC simplifies the graphics system to some extent to make it easier to use ; for example, the Junior can have up to three bitmaps, but only one is supported using BASIC commands.

\section{Graphics Modifiers and Actions}
Following drawing commands PLOT, LINE, RECT, CIRCLE, SPRITE, CHAR and IMAGE there are actions and modifiers which either change or cause the command to be done (e.g. draw the line, draw the string etc.). 
Changes persist, so if you set COLOUR 3 or SOLID it will apply to all subsequent draws until you change it.
Not all things work or make sense for all commands ; you can’t change the dimensions of a line, or the colour of a hardware sprite.

\begin{tabular} {|p{1.0in}|p{4.5in}|}
\hline
to 100,100 & Draws the object from the current point to the new point, or at the new point \\
\hline
from 10,10 & Sets the current point, but doesn’t draw. So you have RECT 10,10 TO 100,100. Note that PLOT requires TO to do something, which is a little odd but consistent. The FROM is optional, but must be used where a number precedes the coordinates (e.g. you can’t do COLOUR 5 100,200 \\
\hline
here & Same as TO but done at the current point \\
\hline
by 4,5 & Same as TO but offset from the current point by 4 horizontal, 5 vertical \\
\hline
solid & Causes shapes to be filled in\\
\hline
outline & Causes shapes to be drawn in outline\\
\hline
dim 3 & Sets the size of scalable objects (CHAR, IMAGE) from 1 to 8.\\
\hline
colour 4 color 5 & Synonyms due to American misspelling, sets the current drawing colour from LUT 0, which is set up as  RRRGGGBB. \\
\hline
\end{tabular}

\section{Some useful examples}

All these examples begin "bitmap on:cls:bitmap clear 3" ; display and clear the bitmap, then fill it with colour 3 - this is 0000 0011 - and the colour by default is RRRG GGBB in binary - so this is Blue

\example{Some lines}
\begin{verbatim}
	100 bitmap on:cls:bitmap clear 3
	110 line colour $1E from 10,10 to 100,200 to 200,50 to 10,10 by 0,20	
\end{verbatim}

Note how you can chain commands  and also the use of the relevant position "by" which means from here.

\example{Some circles}
\begin{verbatim}
	100 bitmap on:cls:bitmap clear 3
	110 circle solid colour $1E outline 10,10 to 200,200 solid 10,10 to 30,30
\end{verbatim}


Rectangles are the same. Currently we cannot draw ellipses.

\example{Some text}
\begin{verbatim}
	100 bitmap on:cls:bitmap clear 3
	110 text "Hello there" dim 1 colour $FC to 10,10 dim 3 to 10,40	
\end{verbatim}

Drawing text from the font library in Vicky. These characters can be redefined.

\example{Some pixels}
\begin{verbatim}
	100 bitmap on:cls:bitmap clear 3
	110 repeat
	120 plot color random(256) to random(320),random(230)
	130 until false
\end{verbatim}

Press break to stop this one. Note you have to write "Plot To" here.

\section{Ranges}
The range of values for draw commands is 0–319 and normally 0–239, though there is a VGA mode which is 320x200 (in which case it would be 0-199).

Colours are values from 0-255 - initially this can be viewed as a binary number RRRG GGBB
