\chapter{Writing Programs in SuperBASIC}

\section {Writing programs}
Programs in SuperBASIC are written in the 'classic' style, using line numbers. A line number on its own deletes a line.\\

LIST operates as in most other systems, except there is the option to list <procedure>() which lists the given procedure by name. (LIST also uses commas, not - as some BASICs do e.g. list 100,300)\\

It is easy to cross develop in SuperBASIC, writing a program on your favourite text editor, and squirting it down the USB cable using a Python program , or the Foenix IDE.\\

Programs can be loaded or saved to SDCard or to a CBM IEC type drive (the 5 pin DIN serial port) using the SAVE and LOAD command.\\

There is also currently a VERIFY command whose purpose it is to check files have been saved correctly. While the SDCard and IEC code has seemed to be reliable in practice, the code is still relatively new ; so when saving BASIC programs in development, I recomment saving them under incremental names (e.g. prog1,bas, prog2.bas) , verifying them, and periodically backing up your SD card.\\

This obviously seems slightly long winded, but is a good defensive measure as there may be bugs in the kernel routines, or the BASIC routines which handle program editing. 
